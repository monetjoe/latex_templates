
%!TeX program = xelatex
%\documentclass[]{ctexart}
\documentclass[a4paper,scheme=chinese,linespread=1.6]{ctexbook}
%\usepackage[a4paper, left=1.27cm, right=1.27cm, top=1.27cm, bottom=1.27cm]{geometry}
\usepackage[a4paper, left=3.5cm, right=3.5cm, top=3.18cm, bottom=3.18cm]{geometry}
\usepackage{ctex}
\usepackage{subcaption}
\usepackage{amsmath}
\usepackage{amssymb}
\usepackage{graphicx}
\usepackage{xeCJK}
\usepackage{xeCJKfntef}
\usepackage{zhnumber}
\usepackage{enumitem}
%\usepackage{parskip}

%\setCJKmainfont[BoldFont={方正悠黑_GBK},ItalicFont={AR PL UKai CN}]{FandolSong}%{Noto Serif CJK SC}%
%\setCJKmainfont{SimSun}%[BoldFont={方正悠黑_GBK},ItalicFont={AR PL UKai CN}]
%\setCJKsansfont{方正悠黑_GBK}

\AtBeginDocument{\addtocontents{toc}{\protect\thispagestyle{empty}}}

\usepackage{multicol}

%\usepackage{titlesec}
%\renewcommand{\chaptername}{第\zhnumber{\thechapter}章}
%\titleformat{chapter}{huge}{第\thechapter章}{1em}{}
%\renewcommand{\thechapter}{\zhnumber{\thechapter}}
%\renewcommand{\contentsname}{目\ \ 录}
%\renewcommand{\listfigurename}{插图目录}
%\renewcommand{\listtablename}{表格目录}
%\renewcommand{\bibname}{参考文献}
%\renewcommand{\abstractname}{摘要}
%\renewcommand{\indexname}{索引}
%\renewcommand{\tablename}{表}
%\renewcommand{\figurename}{图}


\newcommand{\fixeduline}[2][8cm]{\uline{\makebox[#1]{#2}}}
\newcommand{\thetitle}{论文标题}
%\newcommand{\thetitle}{\fontsize{32pt}{32pt}\selectfont 智能混音方法研究}\vspace{1cm}
\newcommand{\engtitle}{English Title}

%\ctexset{
%chapter/format = \LARGE\bfseries\centering,
%section/name = {第,节},
%section/number= \chinese{section},
%subsection/name = {,、},
%subsection/number = \chinese{subsection},
%}

%%%%%%%%%%%%%%%%%%%%%%%%%%%%%%%%%%%%

\begin{document}
\pagestyle{empty}


% TODO 封面信息
\begin{titlepage}
	\begin{center}
		\noindent\includegraphics[width=0.6\textwidth]{media/logo.png}\vspace{2cm}
	\end{center}


	\begin{center}
		{\fontsize{36pt}{36pt}\fontfamily{微软雅黑}\selectfont 博~\:士~\:学~\:位~\:论~\:文}\vspace{2cm}

		{\fontsize{32pt}{32pt}\selectfont\bfseries 标题第一行\protect\\标题第二行 \par}\vfill

		{\LARGE
			作者姓名:\fixeduline{张三}\\
			学  号:\fixeduline{114514}\\
			所在系部:\fixeduline{音乐人工智能与音乐信息科技}\\
			研究方向:\fixeduline{音乐人工智能}\\
			导师姓名:\fixeduline{李四~教授 \quad 王五~教授}\\
			提交时间:\fixeduline{1919年810月}\\
		}
	\end{center}
\end{titlepage}

% TODO 论文原创性声明和使用授权书
{\large%\bfseries

\setlength{\columnsep}{0pt}

{\centering \fontsize{16pt}{16pt}\textbf{中央音乐学院博士学位论文原创性声明}\par}
\vspace{0.3cm}
本人郑重声明:此处所提交的博士学位论文《\thetitle 》,是本人在导师指导下,在中央音乐学院攻读博士学位期间独立进行研究工作所取得的成果。据本人所知,论文中除已注明部分外不包含他人已发表或撰写过的研究成果。对本文的研究工作做出重要贡献的个人和集体,均已在文中以明确方式注明并表示了谢意。本声明的法律结果将完全由本人承担。

\begin{multicols}{2}
	\vfill\null
	\columnbreak
	\textbf{作者签名:}~~~~\includegraphics[width=0.3\columnwidth]{} %括号里是作者电子签
	\par
	\vspace{0.3cm}
\end{multicols}
\hfill \textbf{XXXX~年~XX~月~XX~日}
\vfill

{\centering \fontsize{16pt}{16pt}\textbf{中央音乐学院博士学位论文使用授权书}\par}
\vspace{0.3cm}
《\thetitle 》系本人在中央音乐学院攻读博士学位期间在导师指导下完成的博士学位论文。本人完全了解中央音乐学院关于保存、使用学位论文的规定,同意学校保留并向有关部门送交论文的复印件和电子版本,允许论文被查阅和借阅。

本人授权中央音乐学院,可以将本论文提交中国学术期刊(光盘版)电子杂志社在《中国优秀博硕士学位论文数据库》中发表,可以采用影印、缩印或扫描等复制手段保存学位论文。

\textbf{作者签名: }\includegraphics[width=0.125\columnwidth]{} %括号里是作者电子签
~~~~~~~~~~~~~~~
\textbf{导师签名: }\includegraphics[width=0.125\columnwidth]{}\includegraphics[width=0.15\columnwidth]{} %二位导师签名
\par
\hspace{1cm}

\hfill \textbf{XXXX~年~XX~月~XX~日}
%%%%%%%%%%%%%%%%%%%%%%%%%%%%%%%%%%%%
\pagenumbering{Roman}
\chapter*{摘要}
阿巴阿巴
\pagestyle{plain}





\textbf{关键词}:
巴拉巴拉
\chapter*{Abstract}
blablabla

\textbf{Key Words}:
chipichapa
\tableofcontents
\thispagestyle{empty}

\pagenumbering{arabic}
\pagestyle{plain}

\ctexset{chapter/numbering=true}
\chapter{绪论}
\section{天地玄黄}
好好学习天天向上
\subsection{宇宙洪荒}
bibtex条目是这样引用的\cite{kolasinski2008framework}

图片这样插入
\begin{figure}[ht!]
	\centering
	\includegraphics[width=0.85\columnwidth]{media/spatial_playback_design.png}
	\caption{标签}
	\label{fig:fig1-4}  % 用于交叉引用的标签,自己随便写
\end{figure}

表格这样插入

\begin{table}[htbp]
	\centering
	\begin{tabular}{c|c c c} % 定义表格格式
		\hline
		 &  &  & \\
		\hline
		 &  &  & \\
		 &  &  & \\
		 &  &  & \\
		\hline
	\end{tabular}
	\caption{标签}
	\label{tab:2-1} % 用于交叉引用的标签,自己随便写
\end{table}

\chapter{第二章}
\chapter{第三章}
\chapter{第四章}
\chapter{总结}

%%%%%%%%%%%%%%%%%%%%%%%%%%%%%%%%%%%%%%%%%%%%%
\newpage
\addcontentsline{toc}{chapter}{参考文献}
\bibliographystyle{unsrt}
\bibliography{ref}

\newpage
\chapter*{致谢}
\addcontentsline{toc}{chapter}{致谢}

\end{document}
